\subsection{Main challenges and robot tasks}\label{desc_taref}
The summary of the tasks to be performed for an \textit{in situ} hard coating 
application are: 1) Blade damage inspection for repair and coating; 2) Repair; 3)
System mounting ; 4) Abrasive blasting; 5) Hydraulic profile modeling; 6)
Calibration; 7) HVOF coating.
Tasks 1 to 4 can be performed manually and tasks 2, 4, 5, 6 and 7 by robot. 

%Das tarefas a serem relizadas, são destacadas as seguintes:
%Tarefas que podem ser executadas manualmente:
%\begin{itemize}
 % \item Blade damage inspection, for repair and for coating.%Inspeção e
  % análise de danos na pá, tanto para reparo quanto para revestimento.
 % \item Repair.
 % \item System mounting.%Montagem do sistema.
 % \item Surface sandblasting.%Jateamento da superfície.
%\end{itemize}

%Tarefas que poderão ser executadas
% pelo robô:
%\begin{itemize}
%  \item Hydraulic profile modeling.%Modelar o perfil hidráulico.
%  \item Calibration.
%  \item Sandblasting.
%  \item Repair.
%  \item HVOF coating.
%\end{itemize}

A major problem for \textit{in situ} robotic process is accessibility and
placement. The robot must be brought to the turbine, operate in a confined
and curved space, and has a stiff mechanical base for hard coating accuracy.
%This subsection describes the basic tasks of the robot for \textit{in situ}
%turbine coating. %Generally, the robot should be able to perform the coating
% task as if the blade were not installed in the tubina, as it is done by Rijeza
%company, in an autonomous way. 
The runner's blade should conform to the template, hydraulic profile, before
the coating process. Therefore, the system should build an hydraulic
profile (mapping) and analyze flaws. Also, the system should be calibrated, as
the blade and robot positions are not known.
%Esta subseção descreve as tarefas básicas do robô para o revestimento de
%turbinas \textit{in situ}. Em linhas gerais, o robô a ser desenvolvido deve ser
%capaz de realizar a tarefa de revestimento tal qual seria feita caso a pá não
% estivesse instalada na tubina e de uma maneira autônoma. A pá, antes de ser submetida ao
%processo de revestimento, deve estar em conformidade com o gabarito, perfil
% hidráulico de uma pá intacta. Portanto, uma tarefa do robô é realizar o mapeamento do perfil
%hidráulico, construir um modelo 3D e analisar imperfeições.

In case of deep blade deformations, % caused by cavitation and abrasion, 
a repair by welding should be done. % manually or automatically. 
An operator can manually perform the welding because there is no hard
restrictions as the coating process (accuracy, speed, load). However, the
hostile and confined environment can hinder the manual execution, thus welding
can also be a robot task.
%Em caso de deformações, causados por cavitação e abrasão, estas precisam
%ser removidas manualmente ou de forma automatizada, possivelmente por
%soldagem. A tarefa de soldagem pode
%ser realizada por operador, manualmente, por não possuir todas as restrições
%da tarefa de revestimento (velocidade, precisão, carga e etc), porém o ambiente
%pode dificultar a operação de forma que a execução por um robô seja
%indispensável. 

If the blades conforms to the hydraulic profile, the coating erosion
identification is made measuring the thickness of specific points on the
surface of the blade. An operator with a specific device can manually do this
process, efficiently, in ten minutes. %The coating erosion identification is not
%a robot task.
%Após as pás estarem de acordo com o gabarito, faz-se a
%identificação do desgaste do revestimento, medindo sua espessura em pontos
%pontos específicos sobre a superfície da pá. Manualmente esse
%processo é realizado eficientemente em 10 min, justificando a não necessidade
% de esta ser uma tarefa do robô. 

The abrasive blasting process is generally done before the coating
operation. %, subsection~\ref{sandblasting}.
Rijeza company performs the abrasive blasting manually, but there are studies
and companies performing abrasive blasting with robots \citep{ren2008path}.
%Thus sandblasting could be a robot task.
%Em caso de necessidade de aplicação
%de novo revestimento, é necessária a remoção do revestimento antigo por
%jateamento, a fim de deixar a superfície rugosa e aumentar sua aderência. A
%tarefa de jateamento é atualmente realizada maualmente, mas também pode ser
%realizada pelo robô. Como ambos os lados da pá são revestidos, o jateamento
% deve ser realizado em ambos os lados. Vale ressaltar que, em teoria, pode-se aplicar revestimento por metalização sem retirar o último revestimento,
%porém esse processo ainda se encontra em fase de estudos na Rijeza.
%Segue-se o exemplo de empresas de aviação, onde existe a
%prática de retirar todo o revestimento antigo antes de aplicar o novo.

%Por fim, o robô deverá aplicar o revestimento como 
%forma de prevenir o dano causado pelos fenômenos abrasivos. O robô projetado
%para fazer o revestimento precisa preencher todos os requisitos discutidos na
%subseção~\ref{sec::desc_hvof} e ser adaptável ao ambiente, cujos as restrições
%são discutidos na subseção~\ref{sec::desc_contex}. 

