The design of a robotic system for \textit{in situ} HVOF process is the study
of the simplest conceptual solutions in section~\ref{sec:sota} and
system adaptations for the two following access:
top hatch and a robotic system with small size industrial manipulator and custom
base; bottom hatch and a robotic system with medium size industrial manipulator
and magnetic fixed base.

%O estudo de viabilidade consiste em avaliar as soluções conceituas mais simples 
%da seção~\ref{sec:projeto}, ou seja, um estudo técnico específico para as
%seguintes soluções: acesso pela escotilha superior com manipulador industrial
% de pequeno porte e base customizada operada eletronicamente; acesso pela escotilha
%inferior com manipulador industrial de médio porte e base fixa magnética; e
%acesso pela jusante com manipulador industrial de grande porte e base fixa
%magnética.

The first stage of the system design study is the market survey of
industrial manipulators, regarding the HVOF process requirements
(speed and payload), manipulator workspace, and robot dimensions and weight for
access compatibility. The survey includes the following companies: Yaskawa,
Fanuc, Adept, ABB, and Kuka. The results show that for the top hatch, due to the small
size, the LBR 820 Kuka manipulator is the unique solution that meets the
requirements. For the bottom hatch, there are several manipulators for
the solution.

%A primeira etapa do estudo de viabilidade consiste em pesquisa de mercado por
%manipuladores industriais, levando em consideração os requisitos do processo de
%metalização (velocidade e payload), espaço de trabalho, e as dimensões e peso
%para compatibilidade com o acesso. O resultado da pesquisa mostrou que, em
%relação ao acesso pela escotilha superior, as dimensões reduzidas restringiram
% muito a busca e apenas o manipulador LBR 820 da Kuka satisfaz aos requisitos. Para os outros acessos, há
%variadas soluções de manipuladores industriais.


The other steps are technical evaluations and are divided into the following
subsections: geometric study to confirm reach of the handler at all
the blade points; building the 3D environment in SolidWorks, bases design
mechanical, handling and logistics of access; simulation of the workspace and
and study manipulability; Study sacrifice cards, actuated mechanisms
and to stop the coating material.

%As outras etapas são avaliações técnicas e são divididas nas seguintes
%subseções: estudo geométrico para confirmar alcance do manipulador em todos os
%pontos da pá; construção do ambiente 3D em SolidWorks, projeto de bases
%mecânicas, movimentação e logística de acesso; simulação do espaço de trabalho
% e e estudo de manipulabilidade; estudo de placas de sacrifício, mecanismos atuados
%para interromper o revestimento e materiais. 
