\subsubsection{Projeto de robôs em trilhos}\label{proj_rail}
 % attach a rail to the blade and move it manually
 
 % attach a rail one the nose and ground, 1D movement and move the blade to
A utlização de um manipulador robótico sobre trilhos tem a
capacidade de satisfazer todos os requisitos, até agora observados, para a realização de um
processo de inspeção e metalização utilizando a técnica HVOF. O desenvolvimento
de um sistema compacto para o transporte através dos dutos de acesso e instalção
no camara da turbina é possível, pois o tamanho necessário do manipulador pode
ser reduzido por meio da mobilidade extra proporcionada pela introdução do
trilho.

No contexto da aplicação proposta foram concebidas duas possibilidades para a
fixação do sistema de trilhos. A primeira solução consiste em um sistema
semelhante ao Roboturb, apresentado na seção \ref{sec::rail}. O sistema proposto
se trata de um manipulador robótico, com fixação diretamente na própria pá da
turbina. \textbf{O tamanho do manipulador tem relação direta com o tamanho e
peso do mecanismo de metalização. Outro fator determinante para a viabilidade da
solução é o espaço de trabalho disponível e a distância que o efetuador deve
manter da superfície da pá.}

O trilho deve ser flexível para ser capaz de acompanhar a curvatura da pá e
possibilitar diversas opções de posicionamento. Imãs permanentes podem ser
utilizados como solução de fixação. entrentanto \textbf{deve-se verificar a
resistência do material da superfície da pá, pois dependendo da força magnética
necessária para aguentar o peso do sistema, é possível que o processo de
acoplamento e retirada dos imãs danifique a superfície}. A utilização de
ventosas passivas também pode ser adotada, porém devido a própria natureza desse
sistema é necessário que se desenvolva um sistema de segurança para a detecção
de uma possível perda de poder de sucção. Aceleromêtros podem ser empregados
para verificar deslocamentos e desativar completamente o processo de
metalização. Uma abordagem de segurança menos passiva pode consistir no estudo
do emprego de ventosas ativas e um sistema de controle para reduzi a chance do
robô chegar a se desprender completamente das pás.

\textbf{Caso seja necessária a automatização do processo de preparação e
inspeção da pá}, o efetuador do manipulador deverá ter a possibilidade de troca 
de ferramentas. O manipulador deve ser capaz de suportar o peso de todos os
tipos de sistema e também capaz de seguir as trajetórias específicas para cada
aplicação, assim como, manter a velocidade mínima necessária de cada uma delas.
O tamanho do manipulador e o número mínimo de juntas está relacionado, então,
com um estudo sobre \textbf{as especificações de cada processo a ser realizado
pelo robô, ou seja, as forças relacionadas a cada processo, a complexidade das
trajetórias e as restrições de movimentos impostas pelo método aplicado e,
também, pelo ambiente.}

Todovia, é necessário, para o sucesso da solução baseada em um sistema de
fixação diretamente na pá, que \textbf{após o processo de metalização, seja 
possível a fixação em um ponto recém processado. Caso haja um tempo elevado de 
resfriamento ou fixação do material,
essa solução se torna impossibilitada uma vez que para a total cobertura da 
superfície da pá é necessário que a posição do trilho seja alterada pelo menos 
uma vez.} Outro fator importante de ser observado para essa solução é a
obrigatoriedade da instalação e retirada manual para cada pá da turbina a ser
processada, impossibilitando a manutenção de toda a turbina com uma única
instalação.

Existe uma alternativa, afim de se evitar o contato com pá e, também, a
necessidade de fixação e remoção do robô para cada pá, que consiste em um único trilho
retilíneo fixado no cone do rotor e nas paredes do anel de descarga. Esse tipo
de abordagem simplifica a movimentação do robô no trilho, uma vez que o trilho
seria totalmente reto e possibilitaria a manuntenção de todas as pás com uma
única instalação, \textbf{já que a turbina pode ser rotacionada} de maneira que o robô
esteja posicionado para processar a próxima pá. Entretanto, a movimentação do 
cone influencia diretamente na maneira em que otrilho pode ser acoplado ao 
mesmo. Se o cone girar juntamente com o rotor, umsistema móvel deverá ser 
idealizado. A utilização de rodas, por exemplo,possibilataria que o cone 
girasse sem alterar o posicionamento do trilho. 

\textbf{A curvatura da pá e a distância máxima de sua superfície até a base do
cone} são fatores que podem desfavorecer a escolha dessa solução, já que se o
alcance do robô necessitar ser da mesma ordem de grandeza do comprimento máximo
da pá, o desenvolvimento desse manipulador não teria vantagem sobre um
manipulador robótico comum com uma base fixa no chão. \textbf{A possibilidade
de girar a pá e extensão desse movimento} podem reduzir esse problema, colocando
a superfície da pá o mais pararelo possível ao trilho.

Para ambos os sistemas propostos, é necessário a implementação de um sistema de
localização do robô em relação à pá. Uma vez toda a extensão da pá necessite ser
processada, o robô deverá ter conhecimento de sua localização, para poder,
então, calcular a sua trajetória. O sistema de localização pode ser concebido
tanto com a utilização de equipamentos externos ao robô, quanto instalados no
próprio manipulador ou em sua base. Entre as opções de possíveis soluções se
encontram a utilizam de servo visão e a utilização de laser scanners.

