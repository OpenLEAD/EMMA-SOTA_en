\subsection{The HVOF coating process}\label{sec::desc_hvof}
The thermal spraying (or metallizing) is a process in which
heated materials are sprayed onto a surface in order to improve or
restore their properties and dimensions. The coating extends the material life
cycle, significantly increasing its resistance to erosion and/or corrosion. 
%The different types of thermal spraying are: plasma spraying, detonation
%spraying, wire arc spraying, flame spraying, High velocity oxy-fuel coating
%spraying (HVOF), warm spraying and cold spraying.

%O revestimento por aspersão térmica (ou metalização) é um processo em que
%partículas aquecidas são pulverizadas em uma superfície a fim de melhorar ou
%restaurar suas propriedades e dimensões. O revestimento estende a vida útil do
%material, aumentando significantemente a sua resistência à erosão e corrosão.
%Os diferentes tipos de metalização são: por chama, arco elétrico, detonação,
%chama de alta velocidade (HVOF), plasma, a frio e a quente.

A thermal spraying system comprises: a spray gun, responsible
by partially melting and accelerating the particles to be deposited onto metal
surface; a feeder, which provides the powder (particles) via pipes;
a provider of burning material; a robot (manipulator) to handle the gun; an
electric power supply to the gun; and a control console for the system.
%Um sistema de metalização é composto por: uma pistola de aspersão, responsável
%pelo derretimento e aceleração das partículas a serem depositadas na
%superfície; um alimentador, que fornece o pó (partículas) através de tubos;
%um fornecedor do material de combustão; um robô para manipular a pistola; uma
%fonte de alimentação elétrica para a pistola; um console de controle para o
%sistema.

In the specific case of the Jirau, turbine blades (stainless steel 420) %,
%the thermal spraying type is HVOF on both sides of the blade, and executed 
are coated on High Velocity Oxy-Fuel (HVOF) thermal spraying type by the
Rijeza company with a robotic manipulator with 150 kg payload, which is a good
safety margin, since the system mass is 10 kg (cables and gun). The process
takes 6 hours per side of the blade.
%No caso específico das pás (aço inox 420) das turbinas da usina hidrelétrica de
%Jirau, antes da montagem da turbina, a metalização tipo HVOF é realizada em
%ambos os lados da pá pela empresa Rijeza com um manipulador industrial de 150
% kg de carga máxima, permitindo controle de vibrações com boa margem de segurança, já que a massa do
%sistema pode chegar a 10 kg (cabos e pistola). O tempo
%mínimo do processo é de 6 horas por lado da pá.

The HVOF consists of feeding, in a combustion chamber, the coating material
(tungsten carbide), a gaseous fuel mixture (propane), and oxygen. According to
the data provided by the Rijeza company, the 8 kg spray gun projects a flame of
$3000^oC$, spraying the particles with 700 to 1000 m/s speed, and generating a
15 N recoil force.
%O HVOF consiste em alimentar, numa câmara de combustão, o material de
%revestimento (carboneto de tungstênio), uma mistura gasosa do combustível
% (propano) e oxigênio. De acordo com os dados fornecidos pela empresa Rijeza, a pistola de 8
%Kg projeta uma chama de $3000^oC$, que pulveriza as partículas com velocidade
% de 700 a 1000 m/s, gerando uma força de recuo de 15 N.

The robotic manipulator must have 5 mm accuracy, and the spray gun should remain
at a 230 to 240 mm distance with $90^o\pm 60^o$ angle in respect to the metallic
surface plane. The end effector of the manipulator must control the spray
gun at a constant 40 m/min speed and not stop during the process with the blade
in its range (\textit{long stop}), otherwise coating material would
accumulate.
The end effector direction changes are considered \textit{long stop} too, thus
direction changes should be made out of the blade range, or sacrifice plates
should be used. Sacrifice plates, or masking, are metal plates placed on blade
spots where should not be coated. It is, usually, a plate of common steel as the
flame from the spray gun does not stand still for a long period, not heating it
enough to damage. 
%O manipulador robótico deve possuir precisão de 5 mm, a pistola no efetuador
%deve permanecer a uma distância que varia entre 230 e 240 mm, e ângulo de $90^o
%\pm 60^o$, em relação à superfície. O manipulador deve ser capaz de
%mover a pistola a velocidade constante de 40 m/min, e não pode permanecer uma
%posição da pá por muito tempo (parada), pois há acúmulo de material, deformando
% a superfície. Trocas de direção ou sentido na movimentação do manipulador são
%considerados como parada, logo as trocas deverão ser realizadas em áreas
%exteriores à superfície da pá ou chapas de sacrifício são utilizadas. 
%Placas de sacrifício, ou mascaramento, são chapas colocadas em regões onde as
%peça não podem ser jateadas ou revestidas. Geralmente uma chapa de qualquer
% tipo de aço pode ser utilizada, pois a chama não fica parada sobre ela por um longo
%período, não aquecendo-a o suficiente para danificar. Quando a pistola
%permanece, em funcionamento, a chama é apontada para algum lugar onde não tenha
% obstáculos.

 % As informações do processo
% podem ser observadas na figura~\ref{fig::hvof}.
 
%\begin{figure}[h!]	
%	\includegraphics[width=\columnwidth]{figs/intro/hvof.pdf}
%	\caption{Foto do efetuador do manipulador e pistola HVOF.}
%	\label{fig::hvof}
%\end{figure}

Regarding the operating conditions, the hydropower turbine is a confined
space, the HVOF process has excessive audible noise (100-140 dB), hazardous
gases and potentially explosive gases are expelled; the blade can reach up to
$110^oC$; the environment temperature and humidity should be monitored and
ideally set for the coating process; and $40\%$ of the sprayed particles are
lost during the process \citep{wu2006rebound}, which are spread throughout
the environment. Therefore, some measures must be taken for the execution: the
operation must be teleoperated; no human presence can be required; environment
gases, humidity and temperature must be constantly monitored and controlled; the
robotic manipulator should be sealed; %the wasted particles should be removed afer process (cleaning); 
and system electric/electronic shutdown must be accompanied by gas cutting.
%Em relação às condições de operação: o espaço da aplicação HVOF é confinado,
%com excesso (100 a 140 dB), gases nocivos e com risco de explosão podem
%ser exalados; a pá pode atingir temperaturas de até $110^oC$; as condições de
%umidade e temperatura devem ser ideais para o processo; e há perda de $40\%$
%das partículas pulverizadas  \citep{wu2006rebound}, que são espalhadas pelo
%ambiente. Portanto, algumas medidas devem ser tomadas para a execução do
%processo: a operação deve ser remota, não há presença de pessoas \textit{in
%loco}; os gases presentes e umidade/temperatura devem ser constantemente
%monitorados; o robô manipulador é selado; as partículas desperdiçadas devem
%ser removidas (limpeza); e o desligamento do sistema deve ser acompanhado por
% corte de gás.

%Finally, the coating quality is evaluated by an instrument which measures
%surface porosity, oxidation , hardness and roughness. The measurement is
%performed manually, quickly and easily by an operator.
%A qualidade do revestimento é geralmente avaliada por um instrumento que
%realiza a medida de porosidade, oxidação, dureza e rugosidade da superfície. O
%processo é realizado manualmente, de maneira rápida e fácil, por um operador.


The table~\ref{tab::hvof} summarizes the project restrictions and specifications:
%A tabela~\ref{tab::hvof} resume as restrições e especificações do
%projeto:

\begin{center}
\begin{tabular}{  c | c  }
  \hline
  \textbf{Component} & \textbf{Data} \\ \hline
  Spray gun mass & 8 Kg  \\ \hline
  Cables mass & 12 Kg  \\ \hline
 % Processing time per blade & 12 hours \\ \hline
  Flame temperature & $3000^oC$ \\ \hline
  Spray gun recoil & 15 N \\ \hline
  Manipulator precision & 5 mm \\ \hline
  Spray gun to blade distance & 230-240 mm \\ \hline
  Spray gun to blade angle & $30^o$-$90^o$ \\ \hline
  Manipulator velocity & 40 m/min \\ \hline
  HVOF sound noise & 100 a 140 dB \\ \hline
  Blade temperature & up to $110^oC$ \\
  \hline
\end{tabular}
\captionof{table}{HVOF process data}
%\caption{Dados principais do processo de metalização HVOF}
\label{tab::hvof}
\end{center}


%\begin{center}
%\begin{tabular}{  c | c  }
%  \hline
%  \textbf{Componente} & \textbf{Dado} \\ \hline
%  Massa da pistola HVOF & 8 Kg  \\ \hline
%  Massa dos cabos HVOF & 12 Kg  \\ \hline
%  Tempo HVOF por pá & 6 horas \\ \hline
%  Temperatura da chama HVOF & $3000^oC$ \\ \hline
%  Recuo da pistola & 15 N \\ \hline
%  Precisão do manipulador & 5 mm \\ \hline
%  Distância pistola-pá & 230-240 mm \\ \hline
%  Ângulo pistola-pá & $30^o$-$90^o$ \\ \hline
%  Velocidade do manipulador & 40 m/s \\ \hline
%  Ruído HVOF & 100 a 140 dB \\ \hline
%  Temperatura da pá & $110^oC$ \\
%  \hline
%\end{tabular}
%\captionof{table}{Dados principais do processo HVOF}
%\caption{Dados principais do processo de metalização HVOF}
%\label{tab::hvof}
%\end{center}

%Sistemas robóticos não devem utilizar magnetismo como meio de aderência, já que
%o aço inox 420 não apresenta alta permeabilidade magnética e a alta temperatura
%da pá deve inviabilizar essa solução. Adesão por ventosas é uma solução
%viável, pois material não causa dano ao revestimento, porém a escolha do
%material da ventosa deve ser estudado,já que a pá quente pode ocasionar em
%perda de sucção, como em ventosas emborrachadas.

