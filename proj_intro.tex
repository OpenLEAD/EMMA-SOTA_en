The autonomous system design for HVOF in hydraulic turbine blades include
solutions that meet all the application's requirements. Thus, the envisioned
robots of this section merge some technologies exhibited in
section~\ref{sec:sota} in the context of the Jirau hydroelectric dam.

%O projeto de robôs autônomos para HVOF em pás de turbinas hidráulicas contempla
%as soluções que atendem a \textbf{todos} os requisitos da aplicação. Dessa
%forma, serão idealizados robôs com a fusão das tecnologias expostas na
%seção~\ref{sota} e no contexto da usina hidrelétrica de Jirau. 

In section~\ref{sec::consideracoes}, the arc chamber accesses were described
and their restrictions are essential for the elaboration of the solution.
This section is divided into robotic solutions for both accesses, since they
are the most important development restriction, as they limit robot's
dimensions, features, and demand different logistics.

%Na seção~\ref{sec::consideracoes}, os acessos ao aro câmara foram
%descritos e suas restrições são fundamentais para a elaboração da solução.
%Esta seção é dividida em soluções de sistemas robóticos para os dois tipos
%de acessos, já que estes são o fator que mais restringe o desenvolvimento do
%sistema robótico por limitar suas dimensões, funcionalidades, e exigir a
%idealização conjunta de uma logística de acesso e movimentação do robô pelo aro
%câmara.
