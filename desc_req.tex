\subsection{Requirements for HVOF process}

The HVOF process in hydropower turbines has two prerequisites for correct
coating application: abrasive blasting, and repare. %This subsection details the
%appropriate blades surfaces preparation to ensure the maintenance of the
% hydraulic profile.
%O processo de metalização de turbinas hidrelétricas tem alguns pré-requisitos
%que devem ser respeitados para uma correta aplicação e fixação da camada de
%material durante o revestimento. Essa subseção descreverá as etapas necessárias
%de preparação da superfície a fim de se assegurar a manutenção da qualidade dos
% resultados e do perfil hidráulico da pá. 

%\subsubsection{Sandblasting}\label{sandblasting}

According to Rijeza, recoating an already coated surface does
not produce satisfactory results, thus it is recommended an abrasive blasting
operation to smooth the surface and remove the remaining coating. Abrasive blasting consists in forcibly propelling a stream of abrasive
material against a surface under high pressure. The erosion creates an uniform
superficial layer, increasing the surface roughness and
adhesion.
%O processo de metalização sobreposto a uma superfície que já possui uma camada
%protetora desgastada não apresenta um resultado tão satisfatório se comparado
%com o processo realizado em uma superfície crua. Por esse motivo é recomendado
%que seja realizado um processo de jateamento abrasivo. 

%O jateamento consiste em direcionar um fluxo de material abrasivo na superfície
%do material a fim de se erodir a mesma e retirar o material depositado na
% camada superficial. Outra característica desse processo é a capacidade de aumentar a
%rugosidade da superfície e, assim, aumentar o poder de adesão da nova camada a
%ser metalizada.  

%In the specific case of turbine blades, the sandblasting process utilizes
%aluminum oxide as abrasive material. An operator performs the sandblasting with
%specialized equipment, which is composed a compressed air, and particles
%of the abrasive jet.
%O processo de jateamento para o tratamento específico da superfície das pás da
%turbina utiliza óxido de alumínio como material abrasivo e pode ser realizado
%por um operador. A infraestrutura necessária para esse processo é uma fonte de
%ar comprimido, geralmente proveniente de um compressor de ar, para propulsionar
%o particulado que forma o jato abrasivo. %\textbf{A preparação do ambiente no
% envolto da pá, o escoamento do material e
%as consequências da realização desse processo não foram analisados} e,
%possivelmente, será necessária a implementação de infraestrutura de suporte
%para proteção dos equipamentos adjacentes que não receberão o jateamento,
%limpeza do material depositado e exaustão do particulado suspenso.

%\subsubsection{Repare}

Damages on the blade surface can reduce its efficiency and even its integrity.
The HVOF process can not repair severe surface damage or structural damage,
such as cracks. The repair procedure varies according to the severity of the
damage. Common repair procedures are: 1) non-fused materials; 2) welding; 3)
welding and solid plate.

%Therefore, blade inspection should
% be performed before the HVOF process, since the abrasive blasted surface
% facilitates damage visualization.
%Danos existentes na superfície da pá ou em sua estrutura podem reduzir a sua
%eficiência e até mesmo a própria integridade da pá, prejudicando a segurança 
%da operação. O processo de metalização não tem a capacidade de reparar
%danos severos na superfície ou danos estruturais como rachaduras. A inspeção
%para procura desses defeitos deve ser realizada antes da realização do processo
%de metalização, uma vez que a superfície jateada, ou seja, em metal cru sem
%camada de proteção, facilita a visualização de danos. %Os procedimentos para
%reparos de danos estruturais ou referentes a rachaduras não serão cobertos por
%este documento.
%The damages caused by cavitation, as explained in
%section~\ref{sec::consideracoes}, modifies the blade hydraulic profile and
%should be repaired. 

%Os danos causado por cavitação, como explicado na seção
%\ref{sec::consideracoes}, pode alterar o perfil hidráulico da pá e deve ser
%reparado sempre que possível. O procedimento de reparo varia de acordo com a
%severidade dos danos causados. À medida que a profundidade das cavidades
% geradas na pá e a extensão dos danos vão aumentando, medidas mais extremas se tornam
%necessárias e, por isso, a estratégia de reparo para esse tipo de dano deve
%estar alinhada com o tipo de processo que se deseja utilizar. Inspeções e
%reparos mais frequentes significam processos mais simples, enquanto que reparos
%mais espaçados podem resultar até na inutilização da pá. Os procedimentos mais
%utilizados para o reparo de danos causados por cavitação são:

%\begin{itemize}
%  \item Non fused materials;
%  \item Welding;
%  \item Welding and solid plate.
%\end{itemize}

%\paragraph{Repair with non fused materials}

%Minor damages are usually treated with non fused material processes, where it
% is not necessary to fuse materials to fill the blade cavities. The processes and
%materials used are: epoxy, ceramics, HVOF coating, neoprene and urethane.
%Para pequenos danos, é possível utilizar processos nos quais não é necessário
%fundir o material depositado para preenchimentos das cavidades ao material da
%superfície metálica da pá. Os processos e materiais utilizados, usualmente na
%indústria, são: 

%\begin{itemize}
%\item Epoxy;
%\item Cerâmica;
%\item Revestimento por metalização;
%\item Neoprene;
%\item Urethane.
%\end{itemize}

%Vale ressaltar que a solução proposta para a metalização de uma camada
% protetora para se evitar os danos causados pela cavitação também poderia ser utilizado
%para preencher danos passados, desde que respeitem o limite de espessura
%para o tipo de processo utilizado


%\paragraph{Repair with welding}

%Cavitation repair by welding is the most common procedure, as it allows high
%material deposition and requires less maintenance. This process consists in
%welding layers deposition. After the operation, the repairing surface should be
%polished accordingly with the standard measures of quality for the blade
%hydraulic profile. This procedure is usually performed manually by a highly
%skilled operator. There are, also, in the literature, automated solutions, such
%as the Scompi Robot and Roboturb \citep{roboturb,scompi}.
%O preenchimento dos danos causados devido à cavitação por solda é o
%procedimento mais comum, pois possibilita uma maior deposição de material e não
% obriga a realização de reparos com uma frequência elevada. Esse processo consiste na
%deposição de solda em camadas, até o completo preenchimento. A
%superfície deve ser, então, esmerilhada até entrar em conformidade com as
%medidas padrão de qualidade para o perfil hidráulico da pá a ser reparada. Essa
% tarefa é normalmente realizada por mão de obra altamente qualificada e existe, também,
%na literatura a presença de soluções automatizadas, como os robôs Roboturb e
% Scompi
%\citep{roboturb,scompi}

%\paragraph{Repair by welding and solid plate}

%Severe damages are usually repaired by solid plates, which would fill large
%areas. The fixation process of the plates is carried out by welding, as well
%as filling the remaining cavities. 
%Para casos de danos mais severos, pode ser necessária a utilização de placas
%para o preenchimento de grandes extensões. O processo de fixação das placas é
%realizado por solda, assim como o preenchimento do volume restante. O processo
%de solda, esmerilhamento e verificação é comum ao procedimento padrão
% utilizando somente solda.