Cavitation and abrasion in hydropower turbines cause surface erosion and
blade's hydraulic profile deformation, resulting in efficiency
reduction. A preventive solution is the High Velocity Oxygen Fuel (HVOF) coating
process of the blades. The hard coating creates a lamellar structure,
increasing power generation efficiency, and provides greater
resistance to erosion. In the case of the Jirau hydroelectric dam, the coating
of turbine's blades is performed before turbine assembling and
installation. However, the abrasion due to a large number
of particles and sediment in the Madeira river and the recent identified
cavitation require recoating in short intervals \citep{santa2009slurry}. Turbine
disassembling, recoating and turbine reassembling would be a very expensive
process that should not be done regularly.

%O fenômeno de cavitação e abrasão em hidroturbinas provoca desgaste
%superficial por erosão e alteração do perfil
%hidráulico da pá, gerando redução da eficiência na geração de energia.
%Uma solução preventiva é o revestimento por metalização das pás, o qual aumenta
% a eficiência na geração de energia por gerar uma estrutura mais lamelar, e fornece maior
%resistência a desgastes. No caso da usina hidrelétrica de Jirau, o revestimento
%das pás é realizado antes da montagem e instalação da turbina, porém devido ao
% grande número de partículas e sedimentos que o rio madeira carrega e à cavitação, o revestimento
%deve ser aplicado novamente em intervalos curtos de tempo
%\citep{santa2009slurry}. A desmontagem da turbina, aplicação de novo
%revestimento nas pás e remontagem são um processo muito custoso e deverá ser
%feito regularmente. Portanto, há a necessidade de o procedimento ser
%executado dentro do aro câmara, \textit{in situ}, onde as pás são instaladas.

Cavitation is the formation of vapor cavities (bubbles) in a liquid due to
sudden pressure drops. When the liquid is subjected to increased pressure,
the bubbles implode, causing shock waves \citep{brennen2013cavitation}. 
In hydropower turbines, the cavitation occurs near the blades or
in the turbine output.% The liquid has the combination of kinetic components,
%gravitational potential, and energy flow. The kinetic component is due to the
%water flow (velocity), the potential depends on the altitude of the liquid, and
%the energy flow is the energy that a fluid contains due to pressure. According
%to the Bernoulli principle, the conservation of fluids, it implies that for
%the same altitude, increased kinetic component causes a reduction in pressure,
%occurring cavitation. 
The formation of large bubbles modifies the
characteristics of the flow, causing oscillations or vibrations on the
hydropower turbine, and negative affects the performance of the hydraulic
system. On the other hand, the collapse of small bubbles generates high
frequency shock waves, causing erosions on metal surface.
%A cavitação é a formação de cavidades de vapor (bolhas), em um líquido, devido
% a quedas repentinas de pressão. Quando o líquido é sujeito ao aumento de pressão,
%as bolhas implodem, ocasionando ondas de choque \citep{brennen2013cavitation}.
%Em hidroturbinas, o fenômeno de cavitação é comum próximo às pás ou
%na saída da turbina. O líquido apresenta a combinação
%de componentes cinético, potencial gravitacional e energia de fluxo. O
%componente cinético é em virtude do fluxo da água (velocidade), o potencial tem
%relação com a altitude, e a energia de fluxo é energia que um fluido contém
%devido à pressão que possui. De acordo com o princípio de Bernoulli, o
% princípio da conservação para os fluidos, implica-se que, para uma mesma altitude, o
%aumento da componente cinética acarreta em uma diminuição da pressão, ocorrendo
%cavitação. 
%Quando há cavitação, a formação de bolhas grandes altera as características do
%escoamento, ocasionando oscilações ou vibrações na máquina que, por
%conseqüência, prejudicam o rendimento do sistema hidráulico. As bolhas
%pequenas, ao colapsar, geram ondas de choque de alta frequência, podendo
% provocar erosões se próximo à superfície metálica.

%Além da cavitação, como a água atravessa o aro câmara em grande velocidade, o
%acúmulo de sedimentos irá provocar desgaste abrasivo, isto é, perda de material
%pela passagem de párticulas rígidas. 

In this section, the coating technology is addressed as the way to
reduce the damage by cavitation. Also, the reader is contextualized in
the Jirau hydroelectric plant problem. Finally,  robotic tasks are highlighted to
solve the problem.
%Nesta seção, são apresentadas as formas de reduzir os danos da cavitação pela
%tecnologia de revestimento por metalização, a contextualização do problema no
%caso da usina hidrelétrica de Jirau e as tarefas que um sistema robótico deve
%realizar para solucionar o problema.

