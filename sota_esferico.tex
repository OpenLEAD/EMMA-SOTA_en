\subsection{Manipulator on a fixed spherical base}
An R\&D project was presented in \cite{motta2010prototype} to propose
methodology, simulation and the steps for building a robotic system to repair
hydraulic turbine blades, in highly dangerous environments with $ 40^o C$ to $
99^o C$, and 10 operation hours.


The robot meets the following requirements: ability to operate in any position:
horizontal, vertical, reversed; lightweight for portability and blade fixation;
stiffness to deflection: payload on the wrist occurs in any direction and
extension; high precision; parts availability on the market; user interface;
large workspace; adhesion to the hydraulic turbine blades.


The system has spherical topology and the following characteristics: manipulator
with three Degrees of Freedom (DOF - 2R1P) and wrist with two DOF (2R); 3D
surface mapping with laser scanner; embedded electronics; arc welding gun; blade adhesion by magnetic or
suction devices; low cost; ring-shaped workspace with 2.5 m, and 60 cm height;
30 kg weight and dimensions 30 x 25 x 100 cm; and autonomy.


The robotic manipulator system with spherical base is a compatible solution
for the HVOF application in hydraulic turbine blades, since its
original application is blade repair by arc welding, same environment and
similar challenge. All the system advantages and features are applicable to
the solution of a HVOF system. However, there are particular challenges in the
HVOF process, which are disadvantages of the solution: the HVOF must be
performed in the entire blade, thus the positions of the system must be
manually changed at least 2 times per blade side, and must be moved blade to
blade; the end effector must process the blade with great speed, as required
by the HVOF; complex adhesion system is also needed because blade's temperature
range.
