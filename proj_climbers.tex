\subsubsection{Design of robotic climbers}\label{proj_climbers}
In this subsection, the robotic solutions for HVOF coating are the fusion
of technologies documented in subsection~\ref{sota_climbers}. They will be an
adaptation of \emph{The Climber}, ICM, given their ability to reconfiguration.

%Nesta subseção, consideram-se soluções para HVOF de pás de turbinas robôs
%escaladores com fusão das tecnologias documentadas na
%seção~\ref{sota}, subseção~\ref{sota_climbers}. Será abordada uma versão
%adaptada do robô \emph{The Climber}, ICM, dado sua possibilidade de
%reconfiguração.

\emph{The Climber}, ICM, is a commercial solution which meets many
of the HVOF specifications and enables improvement without compromising its
structure. The robot's adhesion is suction type and the motion is flexible
mats.
The system has already been tested in hazardous environments, as wind turbines,
hydroelectric plants and others. We can divide the design into four systems:
mobility, adhesion, manipulator and autonomy.

%O robô \emph{The Climber}, ICM, é uma solução comercial que atende muitas das
%especificações HVOF e possibilita aperfeiçoamento sem comprometer sua
%estrutura. O robô possui sistema de adesão por sucção e locomoção através de
%esteiras flexíveis. O sistema já foi testado em ambientes de alta
%periculosidade, como turbinas eólicas, usinas hidrelétricas e outros. Podemos
%dividir o projeto em quatro sistemas: locomoção, adesão, manipulador e
% autonomia.

\emph{The Climber} uses only one vacuum chamber instead of the suction cups in
\cite{kim2008development}, for example. \emph{The Climber}'s flexible mats allow
smoothly and continuously motion. The solution with a single chamber seems more
advantageous, as the robot can move on curvatures up to 30 cm radius.

%O sistema desenvolvido em \cite{kim2008development} tem mecanismo de
%locomoção por esteiras e adesão por sucção. O sistema é composto por polias,
%correias de borracha, ventosas, válvulas para cada ventosa, motores DC para as
% polias, sistemas de controle para as válvulas e para os motores. \emph{The Climber}
%utiliza apenas uma câmara de vácuo, em vez de ventosas, e esteiras flexíveis
% que permitem maior suavidade e continuidade ao movimento. A solução por uma única
%câmara parece mais vantajosa, já que o robô consegue se locomover em curvaturas
%de até 30 cm de raio.

In the specific case of the HVOF process, a manipulator applies the coating 
while the robot travels along the blade. The robot locomotion on the blade rises
some design issues: the blade temperature during the procedure requires an
active suction chamber special material; and mats and suction chamber must
work on highly curved surface.

%No caso específico da aplicação HVOF, o processo é realizado com
%manipulador enquanto o robô percorre a pá da turbina. A locomoção do
%robô sob a pá levanta algumas questões de projeto: a
%temperatura da turbina durante o procedimento exige uma solução por câmara
% ativa de material especial; e como se comporta o robô em curvaturas
%acentuadas. 


In adhesion by suction, an intelligent security mechanism should be implemented,
with accelerometers, gyros and other sensors to ensure the shutdown of the
electronics and the supply of the HVOF gases in case of fall. The solution of a
mobile robot path planning increases safe operation and the optimal control of the adhesion
mechanism can limit the maximum suction force.

%Em sistemas de adesão por sucção, deve-se considerar um mecanismo
%inteligente de segurança, possivelmente utilizando acelerômetros e outros
%sensores, para garantir o desligamento do sistema eletrônico e o fornecimento
% de gáses. A solução de um robô móvel com planejamento de trajetória aumenta a
%segurança da operação e o controle ótimo do mecanismo de adesão pode limitar a
% força máxima de sucção.

The manipulator to be designed for HVOF application should have the following
characteristics: to be lightweight, avoiding complex adhesion and balance;
to be fast and accurate as required by the HVOF application; to be modular, as
the operation is performed in confined spaces; to be small, improving mobility,
but sufficient to operate blade edges considering the 230 mm minimum distance
between HVOF gun and blade; and to have high payload and vibration resistance.

%O manipulador a ser projetado para aplicação HVOF possui as seguintes
%características: é leve para não comprometer a adesão e equilíbrio do sistema
%móvel; rápido e preciso conforme requer a aplicação HVOF; modular, já
%que a operação será realizada in-situ, em espaço confinado;
%não possui grandes dimensões, pois o robô é móvel e pode
%percorrer a pá, porém deve ser suficiente para operar em pontos de
%difícil acesso à base e considerar a distância mínima (230 mm) entre pistola
%HVOF e pá; e é capaz de sustentar a carga e vibrações geradas pela
%pistola HVOF. 

The path planning solution should consider both the mobile base and the
manipulator. The literature is fairly consolidated on robotic manipulators, and
many of these problems are already settled and available, as developed in
\cite{manzdevelopment}. The smaller industrial manipulators with required
payload weighs 30 to 50 kg. Therefore, manipulator, HVOF gun and cables weighs
50 to 80 kg.

%A solução de robôs escaladores exige planejamento de trajetórias tanto da base
%móvel, quanto ao controle de manipuladores. A literatura sobre
%manipuladores é bastante consolidada, sendo muitos dos problemas citados já
%resolvidos e disponíveis no mercado, como o desenvolvido em
%\cite{manzdevelopment}. Os menores manipuladores industriais que sustentam a
%carga do sistema de metalização possuem em torno de 30 a 50 kg. Portanto, o
%conjunto manipulador, pistola e cabos pode possuir de 50 a 80 kg de massa.

%A tecnologia que verifica a necessidade de
%revestimento, com sensores laser e ultrassom, e poderá indicar o \textbf{mapa
%ou apenas realizar um teste de sucesso/falha} \citep{escaler2006detection}.

The mission control is the planning and execution of tasks. The motion and
adhesion control should be synchronized, performing the path planning,
obstacle avoidance and environmental mapping, through a set of sensors such as
accelerometers and laser. The manipulator control can be kinematic by visual
servoing or by structured environment. And a vehicle support system will be
responsible for security, smooth operation and power management.

%O sistema autônomo de um robô móvel é a inteligência do robô. Ele abrange o
%controle de missão, ou seja, o planejamento e execução das tarefas em modo
%autônomo. A locomoção será realizada pelo controle dos motores em conjunto com
% o controle do sistema ativo de adesão por sucção, o planejamento de trajetória, desvio de
%obstáculos e mapeamento do ambiente, através de um conjunto de sensores, como
%laser e acelerômetros. O controle do manipulador poderá ser cinemático por
%servovisão ou pela estruturação do ambiente. E um sistema de suporte do veículo
%ficará responsável pela segurança, bom funcionamento e gerenciamento de
% potência do robô.


The climber as described above does not switch automatically between blades.
A climber with arms to switch between blades is a costly solution in terms
of control and mechanical structure. Another solution would be a robot with
locomotion by sliding segments, as RRX3, and adhesion by suction, but the
flexibility required for motion between blades complicates the design. Thus, the exchange between blades should be
manual.

%As características descritas acima como solução de um robô escalador impede a
%troca automática entre pás. Um robô escalador com tecnologia de avanço
% pendurado por braços é uma solução muito custosa em termos de controle e estrutura
%mecânica. Outra solução seria um robô com locomoção por segmentos deslizantes,
%como o RRX3, e adesão por sucção, porém a flexibilidade exigida para a
% locomoção entre pás e a distância entre turbinas complexifica o projeto. Dessa forma, a
%troca entre pás deverá ser manual.

%\textbf{Versão adaptada Roboturb}

%O Roboturb, como já descrito na subseção~\ref{sec::rail}, é um
%manipulador que se locomove em um trilho, este acoplado à pá da turbina
%por ventosas (sucção). A solução não permite a extensão
%do manipulador, já que o peso desequilibra a estrutura e não há torque para
%compensar a força exercida no efetuador durante a operação HVOF. A segunda
%solução de robôs escaladores é adicionar um trilho perpendicular e transformar
%o Roboturb em um robô móvel, com locomoção através de dois trilhos, idéia
%semelhante ao \emph{Climbing robot for Grit Blasting}, que utiliza duas
%plataformas deslizantes com ventosas.

%Os trilhos são compostos por esteiras flexíveis nas extremidades para a
%locomoção, como \emph{The Climber}, e as ventosas são ativas e distribuídas por
%todo o trilho. O manipulador só necessitaria mover em um dos trilhos para
%percorrer toda a pá, já que os trilhos também se movimentam. 

%A solução de trilhos móveis com manipulador é dependente à curvatura da pá da
%turbina e o aumento da flexibilidade do trilho para se locomover sob a pá pode
%impedir a movimentação do manipulador. Dessa forma, é considerada uma solução
%muito específica e restrita à aplicação.

\textbf{Solution conclusion}
Although tempting because of the autonomy, the surface complexity of the
turbine blade, the confined environment, and the required speed and payload are
major challenges to the design. It is estimated a load of 50 kg, which greatly
increases the dimensions of the mobile base and hence diminishes their
workspace, slowing the process.

%Apesar de tentador devido à autonomia, a complexidade da estrutura da pá, o
%ambiente, a velocidade requerida e a carga do sistema de metalização são
%grandes desafios ao projeto. São estimados 50 kg de
%carga para o conjunto manipulador, cabos e pistola, o que aumenta muito as
%dimensões da base móvel e, consequentemente, diminui a sua área de atuação,
%tornando o processo mais demorado.
