\section{Conclusion and future work}\label{sec:conclusions}
The maintenance of hydroelectric turbines is essential for hydropower plants
fully operation, as it substantially increases the power plant potential. The
maintenance of the hydraulic profile of turbine's blades is a major concern for
turbine efficiency, thus regular inspections, repairs and coating application
for cavitation and abrasion protection should be done.

The current hard coating operation is costly, as it requires turbine
disassembling and recalibration. This document aimed to: analyze the
constraints of the \textit{in situ} thermal spray coating process; characterize the environment
where the process is taking place; make a state of the art study
of similar problems; and design conceptual solutions.

%Este documento teve como objetivo: fazer uma análise das restrições do processo
%de revestimento por aspersão térmica; caracterizar o ambiente de trabalho onde
% o processo será realizado; fazer um estudo detalhado do estado da arte que
%visaram solucionar um problema semelhante ou possuíam tecnologias que
%poderiam ser utilizadas como solução; apresentar soluções conceituais; e
%fazer um estudo de viabilidade técnica para as soluções. 

The feasibility study for an \textit{in situ} coating application is promising
and some possible solutions were investigated for each turbine access. As a
conclusion of the proposals, the concept solution is to use an industrial
manipulator on a customized base. The characteristic of the manipulator and the
base varies with the hatch (top or bottom). If top hatch, the solution is a
small sized industrial manipulator and custom telescopic base, electronically
operated; in the case of the bottom hatch, the solution is a mid-sized
industrial manipulator with two rail type based and magnetic holders.

Climbers, systems fixed on the blade, and mobile robots were analyzed.
However, those solutions are not suitable for: the HVOF process, due to speed
and payload requirements; or the geometric complexity and
confined environment of the hydropower turbine, sloping, slippery, high
temperature and humidity. The proposed solutions run into some logistical and
technical challenges, which will be detailed at the development of EMMA project.

The bottom hatch's solution is the most general among the proposed
solutions, as the top hatch is a peculiarity of Jirau's power plant. Thus,
future work, the concept solution for bottom hatch access will be fully
detailed in terms of mechanics, electronics, software and control.



%O estudo de viabilidade de uma solução para revestimento \textit{in situ} se
%mostrou promissor e foram apontadas algumas possíveis soluções considerando
% cada acesso ao aro câmara da turbina. Todas as soluções esbarram em alguns desafios
%logísticos e técnicos que serão abordados detalhadamente até o fim do projeto
%EMMA. Os projetos de bases mecânicas para as diversas soluções serão abordados,
%assim como suas instalações, manuseio e posicionamento. Além disso, toda a
% parte de localização, calibração e mapeamento realizado pelo robô, seu controle e
%interface de usuário ainda serão desenvolvidos.
