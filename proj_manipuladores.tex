\subsubsection{Design of industrial manipulator with rail on
floor}\label{proj_manip}
There are several industrial manipulators which fulfill the HVOF requirements.
The companies Fanuc, Motoman, ABB and KUKA produce manipulators with compatible
dimensions to the bottom hatch; speed, accuracy, and workspace that meet the
requirements for HVOF process; and manipulators which can coat all over one side
of the blade, in a fixed base position. However, manipulators are too big to
operate behind the blade, on the distributor side, due to collision and joint
constraints/robot manipulability. Besides, manipulators with such workspaces are
heavyweight, thus the robot placement and locomotion inside the arc chamber
would be complex.

%Há diversos manipuladores robóticos industriais com as especificações
%necessárias para a realização da tarefa de metalização por HVOF. As empresas
%Fanuc, Motoman, ABB e KUKA fabricam manipuladores com dimensões compatíveis com
% o acesso pela escotilha inferior e velocidade, precisão, e espaço de trabalho que
%cumprem os requisitos para a execução do processo em todo um lado da pá, em uma
%base fixa. Porém, há incompatibilidade atrás da pá e a necessidade de escolher
% a posição correta do manipulador em relação à pá, a fim de maximizar a sua área de trabalho, no ambiente da
%turbina, o que pode restringir os seus movimentos.
%Como as pás podem ser giradas até um ângulo de $14.5^o$, são discutidas as
% ideias de posicionamento do manipulador entre as pás, a fim de executar a operação em ambos os lados da
%pá (um lado de cada pá), e o posicionamento fixo à frente e depois atrás à pá.

The base consists of a transport rail and a positioning rail, both are fixed
in the arc chamber by magnetic coupling and/or welding. The first rail starts at
the bottom hatch entry and goes to the blade, allowing the locomotion of the
robot in the arc chamber. The latter is coupled to the main track and positions
the robot close to the blade, enabling displacement along it. Thus, the
base may be summarized in three joints: prismatic, revolution and prismatic (PRP). As the robot
can not not reach the entire blade, there is still the need of different
vertical positions, which should be manually selected. The blade can coat in
linear or circular motion, and, in both cases, the manipulator will be
responsible for the speed, position and gun orientation. The gun direction
exchange should occur outside the blade, or sacrifice plates should be used, or
valves to redirect the coating particles should be used.

%A base por um trilho de transporte e um de posicionamento. Ambos são fixados no
%aro câmara por acoplamentos magnéticos e/ou solda. O primeiro começa na
%entrada da pá e vai até a pá, permitindo o deslocamento do robô pelo aro
% câmara.
%O último é acoplado ao trilho principal e posiciona o robô próximo a pá, além
% de possibilitar o deslocamento ao longo desta. Dessa forma, a base pode ser
%resumida em três juntas: prismática, revolução e prismática (PRP). Como o robô
%não possui alcance de toda a pá, há, ainda, a necessidade de posições verticais
%diferentes escolhidas manualmente. A pá pode ser processada em movimentos
%circulares ou lineares e, em ambos os casos, o manipulador ficará responsável
% pela velocidade, posição e orientação do processo. A troca de sentido de movimento deverá ocorrer fora da
%pá, ou devem ser utilizadas placas de sacrifício, ou válvulas para
%redirecionamento das partículas de revestimento.

%Esse tipo de abordagem simplifica a movimentação do robô no
%trilho, uma vez que o trilho seria totalmente reto, e possibilitaria a
%metalização de um dos lados das quatro pás com uma única instalação de base.
%Porém, mesmo nesta solução, a altura do trilho deverá ser ajustada três vezes
% para cada lado de pá.

It is also required in this design the implementation of a
localization/calibration system, as the robot should know its location in
relation to the blade for autonomous operation. The localization requires
external sensors, as cameras, or installed on manipulator tip, or base..

%Em ambos os sistemas propostos, é necessária a implementação de um sistema de
%localização do robô em relação à pá, tornando possível a geração de um
%planejamento de trajetórias para o processo de metalização. O sistema de
%localização pode ser concebido por sensores externos
%ao robô (câmeras e outros), ou instalados no próprio manipulador/base.


%The alternative to avoid contact with the blade consists of a single rail
%rectilinear fixed by magnetic bases or welding of the rim in the soil chamber.
% Like the robot has not reach all the shovel, there is still a need positions
%different vertical. The blade can be processed in circular movements or
%linear and, in both cases, the handler will be responsible for the speed,
%position and orientation of the process. The exchange direction of movement
% should occur outside the shovel or must sacrifice cards used.

\textbf{Posicionamento entre pás}

A figura~\ref{fig::andaime} mostra o espaço entre as pás da turbina, dentro do
aro câmara. Um robô manipulador de médio porte pode ser fixado em uma base
magnética, na posição que se encontra a escada da figura~\ref{fig::andaime}.
Essa posição é vantajosa por possibilitar a execução da tarefa em duas pás
(frente de uma e verso da outra), sem desmontar ou fazer grandes alterações no
posicionamento da base do robô, diminuindo as intervenções e tempo de tarefa.

O estudo puramente geométrico demonstra que o alcance do manipulador robótico
para o processamento de ambos os lados das pás, considerando uma base fixa entre
as pás, deverá ser em torno de 5 metros. O manipulador industrial IRB5500,
desenvolvido pela ABB para pintura, possui 3 metros de alcance, porém 180 kg, o que já dificulta ou até impossibilita a
logística de movimentação e posicionamento in-situ. Não foi encontrado um robô
industrial com o alcance necessário e que tivesse as dimensões máximas da
escotilha inferior. 

A solução conceitual de posicionar um manipulador industrial entre as pás deve
avaliar, portanto, todas as configurações necessárias da base (orientações e
posições) para garantir que todo o espaço de trabalho do manipulador mais base
cubra os lados de ambas as pás. O número de configurações e o projeto
mecânico da base são necessários para a viabilização da solução,
uma vez que será possível avaliar as intervenções e complexidades. Bases
autônomas diminuem o número de intervenções e aumentam a precisção do sistema,
porém aumentam a complexidade, o custo devido ao número de sensores e atuadores,
e o peso do sistema, prejudicando a logística.

\textbf{Posicionamento à frente e atrás da pá}
Posicionar de maneira fixa um manipulador com base magnética à frente e atrás da
pá para a metalização é uma solução natural, já que é semelhante à utilizada pela
empresa Rijeza atualmente. Um estudo puramente geométrico, utilizando as
dimensões da pá, mostra que o manipulador deve possuir alcance de 1.7 m e ser
posicionado a uma altura de 1.1 m em relação ao solo. Estudos de espaço de
trabalho, manipulabilidade e colisões devem ser realizados para confirmar o
estudo geométrico.

O posicionamento do sistema à frente ou atrás da pá exige intervenções para
rotação da turbina e para o deslocamento do sistema. Em relação a um
sistema com base autônoma entre as pás, o processo parece mais custoso em
intervenções manuais e mais demorado, porém bem mais simples em termos de
robótica.

\textbf{Conclusão da solução com manipuladores industriais}
A utilização de manipuladores industriais é a mais simples, em termos de
sistemas robóticos, dentre todas as soluções para o acesso pela escotilha
inferior.
Não há projeto mecânico do manipulador, já que este será adquirido em um dos fabricantes citados. As dificuldades mecânicas do projeto serão em relação à logística de posicionamento
e movimentação do robô dentro do aro câmara, e no desenvolvimento de uma base,
que pode ser autônoma. Além disso, o projeto fica responsável pelo controle do
manipulador, processamento de dados que envolvem o HVOF, planejamento de
trajetórias e UI.

Os desafios consistem na construção de uma base rígida e a locomoção dos
equipamentos pelo aro câmara. Este projeto conceitual será uma das frentes para
o estudo de viabilidade.
